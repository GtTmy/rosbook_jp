% !TEX root = ./rosbook_jp.tex
%-------------------------------------------------------------------------------
\chapterimage{chapter_head_1.pdf}
%-------------------------------------------------------------------------------
\chapter*{おわりに}

%-------------------------------------------------------------------------------
\section*{おわりに}

以上で本書は終わりです。この本でROSを全然知らない方もROSの導入からSLAM・Gazebo・MoveItまで体験できると思います。この本で、短時間でROSの専門家になることはないですが、この本を通して「ロボット」の「プログラミング開発環境」を一つ身に付けることはできると思います。また、ROSは単純にロボットプログラミングの開発環境だけではなく全世界のロボット研究者が一つになり、ロボティックスの発達に手を組むこと、またそこに参加することをこの本で知ってもらいたいです。

%-------------------------------------------------------------------------------
\section*{引用について}

本書の一部はOpen Source Robotics Foundation, Inc. (OSRF)が管理しているウィキサイト(http://wiki.ros.org/)を参照しています。このサイトは クリエイティブ・コモンズ 表示 3.0 非移植 ライセンスの下に提供されています。このライセンスのコピーを閲覧するには、http://creativecommons.org/licenses/by/3.0/ を訪問して下さい。他の引用については本文中で表示しております。

%-------------------------------------------------------------------------------
\section*{進化する本}

本書のライセンスは、クリエイティブ・コモンズ・表示 - 非営利 4.0 国際・ライセンス(CC BY-NC)で提供されています。さらに、本の内容や図も以下のGithubレポジトリーですべて公開しております。本書の間違い情報や問題があれば以下のレポジトリーでIssueやPull Requestにて教えていただけると有り難いです。

\begin{lstlisting}
https://github.com/irvs/rosbook_jp
\end{lstlisting}

本書で使ったプログラムも以下のGithubですべて公開しております。間違いや意見などがありましたら、IssueやPull Requestにて教えていただけると有り難いです。

\begin{lstlisting}
https://github.com/irvs/irvs_ros_tutorials
https://github.com/irvs/rosbook_kobuki
https://github.com/irvs/rosbook_robot_arm
\end{lstlisting}

%-------------------------------------------------------------------------------
\section*{本のバージョン}

本書はウェブ上で無料公開するPDFフォマットです。著者の表示し、非営利の目的であれば、許可なして印刷し自由に使えます。ただし、その際は本のバージョンを確認してください。本書の一番最新バージョンは以下のアドレスからダウンロードできます。

\begin{lstlisting}
http://irvs.github.io/rosbook_jp/
\end{lstlisting}

%-------------------------------------------------------------------------------
