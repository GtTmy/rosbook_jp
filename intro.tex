% !TEX root = ./rosbook_jp.tex
%-------------------------------------------------------------------------------
\chapterimage{chapter_head_12.pdf}
%-------------------------------------------------------------------------------
\chapter*{はじめに}

%-------------------------------------------------------------------------------
\section*{ROSの解説書}

本書はROS(Robot Operating System)の解説書です。ROSを全然知らない方もこの本一つでROSの導入からSLAM・Gazebo・MoveItまで体験できます。短時間でロボットプログラミング専門家になることはありませんが、この本を通して「ロボット」の「プログラミング開発環境」を一つ身に付けておくとロボットプログラミングを始めるとき、きっと役に立つことでしょう。また、ROSに触れることは単純にロボットプログラミングの開発環境を知ることだけではなく、あなたがロボット研究者の一員となって、ロボティックスの発展に貢献していることを実感してもらいたい。

%-------------------------------------------------------------------------------
\section*{ライセンス}

本書はクリエイティブ・コモンズ・表示 - 非営利 4.0 国際・ライセンスで提供されています。このライセンスのコピーを閲覧するには、 \url{http://creativecommons.org/licenses/by-nc/4.0/} を訪問して下さい。

%-------------------------------------------------------------------------------
\section*{引用について}

本書の一部はOpen Source Robotics Foundation, Inc. (OSRF)が管理しているウィキサイト(http://wiki.ros.org/)を参照しています。このサイトは クリエイティブ・コモンズ 表示 3.0 非移植 ライセンスの下に提供されています。このライセンスのコピーを閲覧するには、http://creativecommons.org/licenses/by/3.0/ を訪問して下さい。他の引用については本文中で表示しております。

%-------------------------------------------------------------------------------
\section*{進化する本}

本書の内容や図も次のGithubレポジトリーですべて公開しております。本書の間違い等がございまいしたら以下のレポジトリーでIssueやPull Requestにて教えていただけると幸いです。

\begin{lstlisting}
https://github.com/irvs/rosbook_jp
\end{lstlisting}

\noindent 本書で使ったプログラムも以下のGithubですべて公開しております。間違いや意見などがありましたら、IssueやPull Requestにて教えていただけると幸いです。

\begin{lstlisting}
https://github.com/irvs/irvs_ros_tutorials
https://github.com/irvs/rosbook_kobuki
https://github.com/irvs/rosbook_robot_arm
\end{lstlisting}

%-------------------------------------------------------------------------------
\section*{本のバージョン}

本書はウェブ上で無料公開するPDFフォーマットです。著者を表示し、非営利の目的であれば、いつでも自由に使用することができます。ただし、本書のバージョンを確認してください。本書の最新バージョンは以下のアドレスからダウンロードできます。本のバージョンの確認は、PDF1ページ下にバージョンと公開日が記載されております。

\begin{lstlisting}
http://irvs.github.io/rosbook_jp/
\end{lstlisting}

%-------------------------------------------------------------------------------
\section*{日本ROSユーザーコミュニティー}

日本では、東京オープンソースロボティクス協会(TORK)やROS JAPAN Users Group が中心になってセミナーや勉強会を開催し、ROSの普及に努めています。このユーザーコミュニティーも積極的にご利用してください。

\begin{lstlisting}
TORK     http://opensource-robotics.tokyo.jp/
ROS JAPAN Users Group     https://groups.google.com/forum/#!forum/ros-japan-users/
\end{lstlisting}

%-------------------------------------------------------------------------------
